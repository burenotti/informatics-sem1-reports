\documentclass[a4paper]{article}

\input{../preset.tex}


\begin{document}

    % номер практики, номер группы, имя студента, имя преподавателя, должность преподавателя
    \mireatitle{5}{ИКБО-14-22}{Буренин А.А.}{Павлова Е. С.}{ассистент}

    \tableofcontents{}

    \clearpage


    \introduction
    Логическая функция от четырех переменных задана в 16-ичной векторной форме.
    Восстановить таблицу истинности.
    Записать формулы СДНФ и СКНФ.
    Построить комбинационные схемы СДНФ и СКНФ в лабораторном комплексе,
    используя общий логический базис.
    Протестировать работу схем и убедиться в их правильности.
    Подготовить отчет о проделанной работе и защитить её.

    Логическая функция из персонального варианта:
    \[ F(a, b, c, d) = 6c6f_{16} \]


    \implementation

    \subsection{Восстановление таблицы истинности}\label{subsec:table-recovering}
    Преобразовав шестнадцатеричный вектор в двоичную СС, получаем восстановленную таблицу
    истинности для $ F $ (Таблица~\ref{tab:truth}).

    \begin{table}[h]
        \begin{center}
            \begin{tabular}{ | c | c | c | c | c | }
                \hline
                $ a $ & $ b $ & $ c $ & $ d $ & $ F $ \\
                \hline
                0     & 0     & 0     & 0     & 0     \\
                \hline
                0     & 0     & 0     & 1     & 1     \\
                \hline
                0     & 0     & 1     & 0     & 0     \\
                \hline
                0     & 0     & 1     & 1     & 1     \\
                \hline
                0     & 1     & 0     & 0     & 1     \\
                \hline
                0     & 1     & 0     & 1     & 0     \\
                \hline
                0     & 1     & 1     & 0     & 1     \\
                \hline
                0     & 1     & 1     & 1     & 0     \\
                \hline
                1     & 0     & 0     & 0     & 0     \\
                \hline
                1     & 0     & 0     & 1     & 1     \\
                \hline
                1     & 0     & 1     & 0     & 1     \\
                \hline
                1     & 0     & 1     & 1     & 1     \\
                \hline
                1     & 1     & 0     & 0     & 0     \\
                \hline
                1     & 1     & 0     & 1     & 1     \\
                \hline
                1     & 1     & 1     & 0     & 1     \\
                \hline
                1     & 1     & 1     & 1     & 1     \\
                \hline
            \end{tabular}
        \end{center}
        \caption{Таблица истинности}
        \label{tab:truth}
    \end{table}

    \subsection{Формулы СКНФ и СДНФ}\label{subsec:formulas}
    По восстановленной таблице истинности (Таблица~\ref{tab:truth}) запишем формулы СДНФ и СКНФ.
    Для построения формулы СДНФ рассматриваем наборы значений переменных, на которых функция равна единице.
    Переменные, равные нулю, берем с отрицанием, а переменные, равные единице, без отрицания.
    В результате этого мы получаем несколько совершенных конъюнкций, объединенных через дизъюнкцию.
    Это образует формулу СДНФ~\eqref{eq:sdnf}.

    \begin{equation}
        \label{eq:sdnf}
        F_{СДНФ}=\bar a \bar b \bar c d + \bar a \bar b c \bar d + \bar a b \bar c \bar d + \bar a b \bar c d + a \bar b \bar c d + a \bar b c \bar d + ab \bar c \bar d + ab \bar c d + abc \bar d + abcd
    \end{equation}

    Для построения формулы СКНФ рассматриваем наборы значений переменных, на которых функция равна нулю.
    Переменные, равные нулю, берем без отрицания, а переменные, равные единице, с отрицанием.
    В результате чего мы получаем несколько совершенных дизъюнкций, объединенных через конъюнкцию.
    Это образует формулу СКНФ~\eqref{eq:sknf}.


    \begin{equation}
        \label{eq:sknf}
        F_{СКНФ}=(a + b + c + d)\cdot(a + b + \bar c + \bar d)\cdot(a + \bar b + \bar c + d)\cdot(\bar a + b + c + d)\cdot(\bar a + b + \bar c + \bar d)
    \end{equation}

    \subsection{Схемы, реализующие СДНФ и СКНФ в общем логическом базисе}\label{subsec:schemas}
    Построим в лабораторном комплексе комбинационные схемы, реализующие СДНФ (Рисунок~\ref{fig:sdnf}) и СКНФ (Рисунок~\ref{fig:sknf}) рассматриваемой функции в общем логическом базисе.

    \begin{figure}[h]

        \centering
        \includegraphics[width=0.8\textwidth]{sdnf.png}

        \caption{\centering Комбинационная схема, реализующая СДНФ в общем логическом базисе.}
        \label{fig:sdnf}
    \end{figure}

    \begin{figure}[h]
        \centering
        \includegraphics[width=0.8\textwidth]{sknf.png}

        \caption{\centering Комбинационная схема, реализующая СКНФ в общем логическом базисе.}
        \label{fig:sknf}
    \end{figure}


    \conclusion
    По данной логической функции от четырех переменных, заданной в шестнадцатеричной векторной форме
    была восстановлена таблица истинности (Таблица~\ref{tab:truth}).
    По ней были составлены формулы СДНФ~\eqref{eq:sdnf} и СКНФ~\eqref{eq:sknf}.
    Построены комбинационные схемы СДНФ (Рисунок~\ref{fig:sdnf}) и СКНФ (\ref{fig:sknf})
    в лабораторном комплексе с использованием общего логического базиса.
    Протестирована работа схем и их правильность.


    \sources
    \begin{enumerate}
        \item Информатика: Методические указания по выполнению практических
        работ / С.С. Смирнов, Д.А. Карпов - М., МИРЭА - Российский технологический университет, 2020.–102 с.
        \item Лекционный материал / С.С. Смирнов.
    \end{enumerate}

\end{document}
