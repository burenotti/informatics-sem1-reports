\documentclass[a4paper]{article}
\author{Буренин А.А}

\input{../preset.tex}

\begin{document}

    % номер практики, номер группы, имя студента, имя преподавателя, должность преподавателя
    \mireatitle{5}{ИКБО-14-22}{Буренин А.А.}{Карпов Д.А.}{}

    \tableofcontents{}

    \clearpage


    \introduction
    Логическая функция от четырех переменных задана в шестнадцатеричной векторной форме.
    Восстановить таблицу истинности.
    По таблице истинности реализовать в лабораторном комплексе
    логическую функцию на дешифраторах тремя способами:
    \begin{enumerate}
        \item используя дешифратор 4--16 и одну дополнительную схему «или»
        \item используя два дешифратора 3--8 и необходимую дополнительную логику;
        \item используя пять дешифраторов 2--4 и одну дополнительную схему «или».
    \end{enumerate}

    Протестировать работу схем и убедиться в правильности их работы.
    Подготовить отчет о проделанной работе и защитить ее.

    Логическая функция из персонального варианта:
    \[ F(a, b, c, d) = 6c6f_{16} \]


    \implementation

    \subsection{Восстановление таблицы истинности}\label{subsec:table-recovering}
    Преобразовав 16-теричный вектор в двоичную СС, получаем восстановленную таблицу
    истинности для $ F $ (Таблица~\ref{tab:truth}).

    \begin{table}[h]
        \begin{center}
            \begin{tabular}{ | c | c | c | c | c | }
                \hline
                $ a $ & $ b $ & $ c $ & $ d $ & $ F $ \\
                \hline
                0     & 0     & 0     & 0     & 0     \\
                \hline
                0     & 0     & 0     & 1     & 1     \\
                \hline
                0     & 0     & 1     & 0     & 0     \\
                \hline
                0     & 0     & 1     & 1     & 1     \\
                \hline
                0     & 1     & 0     & 0     & 1     \\
                \hline
                0     & 1     & 0     & 1     & 0     \\
                \hline
                0     & 1     & 1     & 0     & 1     \\
                \hline
                0     & 1     & 1     & 1     & 0     \\
                \hline
                1     & 0     & 0     & 0     & 0     \\
                \hline
                1     & 0     & 0     & 1     & 1     \\
                \hline
                1     & 0     & 1     & 0     & 1     \\
                \hline
                1     & 0     & 1     & 1     & 1     \\
                \hline
                1     & 1     & 0     & 0     & 0     \\
                \hline
                1     & 1     & 0     & 1     & 1     \\
                \hline
                1     & 1     & 1     & 0     & 1     \\
                \hline
                1     & 1     & 1     & 1     & 1     \\
                \hline
            \end{tabular}
        \end{center}
        \caption{Таблица истинности}
        \label{tab:truth}
    \end{table}

    \subsection{Реализация логической функции на дешифраторе 4--16}\label{subsec:decoder-4-16}
    Реализуем функцию, используя дешифратор 4--16 и дополнительной схемы “или”.
    Для этого нам необходимо разместить его в лабораторном комплексе и выставить необходимые нам параметры.
    Так как количество адресных входов дешифратора равно количеству аргументов функции,
    мы просто подаем наши переменные на вход дешифратора через объединяющую шину.
    При этом значение переменной $ a $ подается на старший адресный вход, а $ d $ – на младший.
    Далее нужно выбрать те выходы дешифратора, номера которых совпадают с номерами наборов значений аргументов,
    и функция равна единице.
    Объединив эти выходы дизъюнкцией, получаем требуемую реализацию (Рисунок~\ref{fig:decoder-4-16}).

    \begin{figure}[h]
        \centering
        \includegraphics[width=0.8\textwidth]{4-16.png}
        \caption{\centering Реализация логической функции с помощью дешифратора 4--16.}\label{fig:decoder-4-16}

    \end{figure}

    \subsection{Реализация логической функции на дешифраторе 3--8}\label{subsec:decoder-3-8}
    Реализуем функцию, используя два дешифратора 3--8 и необходимую дополнительную логику.
    Для этого нам необходимо подать значение переменной $ a $ на их разрешающие входы,
    причем на первый дешифратор оно будет передаваться через инверсию.
    Это связано с тем, что в первой половине таблицы истинности значение $ a $ равно 0,
    а во второй – 1, что и позволяет разделить изначальный дешифратор 4--16 на два дешифратора 3--8,
    один из которых включается в зависимости от значения переменной $ a $.

    Далее нужно подать значения оставшихся трех переменных на адресные входы дешифратора.
    Значение переменной $ b $ подается на старший вход, а значение переменной $ d $ – на младший.
    В процессе работы на выходах всех дешифраторов будут последовательно возникать единичные значения
    в соответствии с поступающей на адресные входы комбинацией значений переменных.
    У первого дешифратора выберем только те выходы, номера которых соответствуют номерам комбинаций переменных,
    в которых функция равна единице в первой половине таблицы истинности.
    Аналогичную операцию проделываем со вторым дешифратором для второй половиной таблицы.
    После объединения выходов дизъюнкцией получаем требуемую реализацию (Рисунок~\ref{fig:decoder-3-8}).

    \begin{figure}[h]
        \centering
        \includegraphics[width=0.8\textwidth]{3-8.png}

        \caption{\centering Реализация логической функции с помощью двух дешифраторов 3--8.}
        \label{fig:decoder-3-8}
    \end{figure}

    \subsection{Реализация логической функции на дешифраторах 2--4}\label{subsec:decoder-2-4}

    Теперь реализуем логическую функцию, используя дешифраторы 2--4 и дополнительную логику.
    Так как количество адресных входов таких дешифраторов в два раза меньше количества аргументов функции,
    то нам потребуется 4 дешифратора, которые будут называться операционными и один дешифратор,
    который будет ими управлять.
    Он будет называться управляющим.
    Всего будет использовано 5 дешифраторов и 2--4 и одна объединяющая дизъюнкция.
    При такой реализации значения двух старших переменных будут подаваться
    на адресные входы управляющего дешифратора,
    который будет включать один из четырех операционных дешифраторов
    в соответствии с комбинацией значений $ a $ и $ b $.
    Всего таких комбинаций четыре: 00, 01, 10 и 11.
    Выходы управляющего дешифратора подключаются к разрешающим входам операционных.
    Когда значения старших переменных равны 0, на нулевом выходе управляющего дешифратора
    образуется единица, которая подается на разрешающий вход первого операционного дешифратора.
    Аналогично с остальными значениями $ a $ и $ b $.

    Теперь каждый из операционных дешифраторов отвечает за свою двоичную тетраду
    в исходной векторной записи логической функции.
    Выберем у каждого операционного дешифратора выходы, где у двоичной тетрады стоят единицы.
    Объединив эти выходы через дизъюнкцию получаем требуемую реализацию (Рисунок~\ref{fig:decoder-2-4}).

    \begin{figure}[h]
        \centering
        \includegraphics[width=0.8\textwidth]{2-4.png}

        \caption{\centering Реализация логической функции с помощью дешифраторов 2--4.}
        \label{fig:decoder-2-4}
    \end{figure}


    \section{Выводы}\label{sec:conclusion}
    В ходе работы была восстановлена таблица истинности для данной логической функции,
    заданной в шестнадцатеричной векторной форме (Таблица~\ref{tab:truth}).
    По таблице истинности была реализована в лабораторном комплексе логическая функция
    на дешифраторах тремя способами:
    \begin{enumerate}
        \item используя дешифратор 4--16 и одну дополнительную схему «или» (Рисунок~\ref{fig:decoder-4-16});
        \item используя два дешифратора 3--8 и необходимую дополнительную логику (Рисунок~\ref{fig:decoder-3-8});
        \item используя пять дешифраторов 2--4 и одну дополнительную схему «или» (Рисунок~\ref{fig:decoder-2-4}).
    \end{enumerate}


    \section{Информационные источники}\label{sec:sources}
    \begin{enumerate}
        \item Информатика: Методические указания по выполнению практических
        работ / С.С. Смирнов, Д.А. Карпов - М., МИРЭА - Российский технологический университет, 2020.–102 с.
        \item Лекционный материал / С.С. Смирнов.
    \end{enumerate}

\end{document}
